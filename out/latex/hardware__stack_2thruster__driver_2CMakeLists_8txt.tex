\hypertarget{hardware__stack_2thruster__driver_2CMakeLists_8txt}{}\section{src/hammerhead/hardware\+\_\+stack/thruster\+\_\+driver/\+C\+Make\+Lists.txt File Reference}
\label{hardware__stack_2thruster__driver_2CMakeLists_8txt}\index{src/hammerhead/hardware\+\_\+stack/thruster\+\_\+driver/\+C\+Make\+Lists.\+txt@{src/hammerhead/hardware\+\_\+stack/thruster\+\_\+driver/\+C\+Make\+Lists.\+txt}}
